\documentclass[aps,onecolumn,amsmath,amsfonts,amssymb,nofootinbib,eqsecnum,%
  secnumarabic,notitlepage]{revtex4-1}
\usepackage[utf8]{inputenc}
\usepackage{slashed,hyperref,color}
\hypersetup{bookmarks=true, unicode=true, pdftoolbar=true, pdfmenubar=true,
  pdffitwindow=false, pdfstartview={FitH}, pdftitle={DMsimp\_t},
  pdfauthor={Benjamin Fuks}, pdfsubject={FR DMsimp\_t implementation},
  pdfcreator={Benjamin Fuks},pdfproducer={Benjamin Fuks},
  pdfkeywords={dark matter}, pdfnewwindow=true, colorlinks=true, linkcolor=blue,
  citecolor=magenta, filecolor=magenta, urlcolor=cyan}

%%%%%%%%%%%%%% Begin Commands %%%%%%%%%%%%%%%%%%%%%%%%%%%%%%%%%%%%%%%%%%
%% Latin
\def\ie{{\it i.e.}}

%% environments
\newcommand{\be}{\begin{equation*}}
\newcommand{\ee}{\end{equation*}}
\def\bsp#1\esp{\begin{split}#1\end{split}}
\def\bpm{\begin{pmatrix}}
\def\epm{\end{pmatrix}}

%% Physics symbols
\def\lag{{\cal L}}
\def\sss{\scriptscriptstyle}
\def\as{\alpha_{\sss s}}
\def\gs{g_{\sss s}}
\def\gL{g_{\sss w}}
\def\gY{g_{\sss Y}}
\def\gF{G_{\sss F}}
\def\tw{\theta_{\sss W}}
\def\ydm{y_{\sss \chi}}
\def\dR{d_{\sss R}}
\def\eR{\ell_{\sss R}}
\def\Ll{L_{\sss L}}
\def\QL{Q_{\sss L}}
\def\QLbar{\bar Q_{\sss L}}
\def\uR{u_{\sss R}}
\def\uRbar{{\bar u}_{\sss R}}

% Tools
\newcommand{\ch}{{\sc\small CalcHep}}
\newcommand{\fa}{{\sc\small FeynArts}}
\newcommand{\fr}{{\sc \small FeynRules}}
\newcommand{\ma}{{\sc\small MadAnalysis}~5}
\newcommand{\maddm}{{\sc\small MadDM}}
\newcommand{\mg}{{\sc\small MG5\_aMC}}
\newcommand{\micromegas}{{\sc\small MicrOMEGAs}}
\newcommand{\mthmtc}{{\sc\small Mathematica}}
\newcommand{\nloct}{{\sc\small NLOCT}}
\def\lqdm{{\tt LQDM}}
%%%%%%%%%%%%%% End of Commands %%%%%%%%%%%%%%%%%%%%%%%%%%%%%%%%%%%%%%%%%

%%%%%%%%%%%% Begin Cover Page %%%%%%%%%%%%%%%%%%%%%%%%%%%%%%%%%%%%%%%%%%
\begin{document}
\title{Documentation on the \lqdm\ \fr\ implementation}

\author{Benjamin~Fuks}
\email{fuks@lpthe.jussieu.fr}
\affiliation{Sorbonne Universit\'e, CNRS, Laboratoire de Physique Th\'eorique et
  Hautes \'Energies, LPTHE, F-75005 Paris, France}
\affiliation{Institut Universitaire de France, 103 boulevard Saint-Michel,
  75005 Paris, France}

\begin{abstract}
This note contains information about the \lqdm\ model implementation in \fr, a
model aiming to jointly explain flavour anomalies and dark matter. All \fr\
model files are available from
\url{https://phystev.cnrs.fr/wiki/2019:groups:bsm:rdm}, together with an
illustrative \mthmtc\ notebooks and the generated \ch\ and UFO models.
\end{abstract}

\maketitle

%%%%%%%%%%%%%% Begin Main Part %%%%%%%%%%%%%%%%%%%%%%%%%%%%%%%%%%%%%%%%%

%%%%%%%%%%%%%%%%%%%%%%%%%%%%%%%%%%%%
We consider a simplified model in which we supplement the Standard Model by one
scalar leptoquark field $S_1$ and two extra fermionic fields, a Majorana
fermion  $\chi_0$ and a Dirac fermion $\chi_1$. The $S_1$ and $\chi_1$ fields
are electrically-charged coloured weak singlets lying in the
$({\bf 3}, {\bf 1})_{-1/3}$ representation of the Standard Model gauge group. In
contrast, $\chi_0$ is a dark matter candidate and therefore a non-coloured
electroweak singlet. In our \fr\ implementation, we consider all potential
interactions of the new sector with the Standard Model sector, the corresponding
Lagrangian being written as
\be\bsp
 \lag = &\ \lag_{\rm SM} + \lag_{\rm kin}
   + \bigg[
    {\bf \lambda_{\sss R}}\ \uRbar^c\ \eR^{\phantom{c}} \ S_1^\dag
  + {\bf \lambda_{\sss L}}\ \QLbar^c \!\cdot\! \Ll^{\phantom{c}} \ S_1^\dag
  + \ydm \bar\chi_1\chi_0 S_1
   + {\rm h.c.} \bigg]\ .
\esp\label{eq:lag}\ee
In this expression, all flavour indices are understood for simplicity and the
dot stands for the $SU(2)$-invariant product of two fields lying in its
fundamental representation. In addition,
$\lag_{\rm SM}$ is the Standard Model Lagrangian and $\lag_{\rm kin}$ contains
gauge-invariant kinetic and mass terms for all new fields, the $\chi_1$ state
being assumed vector-like. The $\QL$ and $\Ll$ spinors stand for the $SU(2)_L$
doublets of left-handed quarks and leptons respectively, whilst $\uR$ and $\eR$
stand for the $SU(2)_L$ singlets of right-handed down-type quarks and charged
leptons, respectively. As can be derived from the omitted flavour structure, the
${\bf \lambda_{\sss L}}$ and ${\bf \lambda_{\sss R}}$ couplings are $3\times 3$
matrices in the flavour space, that are considered real in the following.
Moreover, in our
conventions, the first index $i$ of any $\lambda_{ij}$ element refers to the
quark generation whilst the second one $j$ refers to the lepton generation.


The field content of the new physics sector of our simplified model is
presented in table~\ref{tab:fields}, together with the corresponding
representation under the gauge and Poincar\'e groups, the potential Majorana
nature of the different particles, the adopted symbol in the \fr\
implementation and the PDG identifier that has been chosen for
each particle. The new physics coupling parameters are collected in
table~\ref{tab:params}, in which we additionally include the name used in the
\fr\ model and the Les Houches blocks in which the numerical values of
the different parameters can be changed by the user when running tools like
\mg\ or \micromegas.

\begin{table}[h]
\renewcommand{\arraystretch}{1.4}
\setlength\tabcolsep{8pt}
\begin{tabular}{c c c c c c}
  Field & Spin & Repr. & Self-conj. & \fr\ name & PDG code\\
  \hline\hline
  $S_1$    & 0   & $({\bf 3}, {\bf 1})_{-1/3}$ & no  & {\tt LQ} & 42\\
  $\chi_0$ & 1/2 & $({\bf 1}, {\bf 1})_0$      & yes & {\tt chi0} & 5000521\\
  $\chi_1$ & 1/2 & $({\bf 3}, {\bf 1})_{-1/3}$ & no  & {\tt chi1} & 5000522\\
\end{tabular}
\caption{\it New particles supplementing the Standard Model, given
  together with the representations under $SU(3)_c\times SU(2)_L \times U(1)_Y$.
  We additionally indicate whether the particles are Majorana particles,
  their name in the \fr\ implementation and the associated Particle Data Group
  (PDG) identifier.}
\label{tab:fields}
\end{table}

\begin{table}[h]
\renewcommand{\arraystretch}{1.4}
\setlength\tabcolsep{15pt}
\begin{tabular}{c c c c}
  Coupling & \fr\ name & Les Houches block\\
  \hline\hline
  $(\lambda_{\sss L})_{ij}$ & {\tt lamL} & {\tt LQLAML}\\
  $(\lambda_{\sss R})_{ij}$ & {\tt lamR} & {\tt LQLAMR}\\
  $\ydm$ & {\tt yDM} & {\tt DMINPUTS}\\
\end{tabular}
\caption{\it Couplings of the new particles, given together with the associated
  \fr\ symbol and the Les Houches block of the parameter card.}
\label{tab:params}
\end{table}

%%%%%%%%%%%%%%%%%%%%%%%%%%%%%%%%%%%%%%%%
\end{document}

